\section{七个未定式}

\subsection{七种未定式}

\begin{LearnningObjectivesBox}[4]
	\begin{learningtargets}
		\item $\frac{0}{0}$ 型
		\item {\small $\dfrac{\infty}{\infty}$}型
		\item {\small ${\infty}-{\infty}$}型
		\item  ${\infty}\cdot{0}$ 型
		\item $1^{\infty} 、0^0 、{\infty^0} $ 型
	\end{learningtargets}
\end{LearnningObjectivesBox}
\begin{KeyandDifficultyBox}[4]
	\begin{learningkeys}
		\item[重点:] 泰勒公式、四则运算、洛必达法则
		\item[难点:] 泰勒公式
	\end{learningkeys}
\end{KeyandDifficultyBox}

\subsubsection{$\frac{0}{0}$ 型}
	\begin{enumerate}
		\item 无穷小相减$\Rightarrow$泰勒公式
		\item 遇见$\sqrt{\Box}$ 先有理化后观察
		\item 四则运算与洛必达法则配合
		\item 变限积分所表示的函数求极限采用洛必达法则\\
			\quan{1} 积分号里面含$x$拆项到积分号外\\
			\quan{2} $f$里含有$x$,需要用变量代换,将其变化到积分限中才可以用积分求导定理
	\end{enumerate}

\subsubsection{$\frac{\infty}{\infty}$型}
	\begin{enumerate}
		\item 当$x \to \infty$时,令$x = -t$然后提取其{\bfseries 最大项}
		\item 分子坟墓为$x$的多项式或简单的根式提出
		\item 影响分子分母趋于无穷大的因子
	\end{enumerate}


\subsubsection{\small $\infty - \infty$型}
	\begin{enumerate}
		\item 提取最大项/通分化为$\frac{0}{0}$型
		\item 倒代换/乘除某式子化为$\frac{0}{0}$型
	\end{enumerate}

\subsubsection{$\infty \cdot 0$型}
	\begin{enumerate}
		\item 当含有$e^x$时,尝试提出化简后在观察形式
		\item 等价无穷小
		\item 极限四则运算法则
		\item 泰勒公式
	\end{enumerate}

	\subsubsection{${1^\infty}、{0^0}、{\infty^0}$型}
	\begin{enumerate}
		\item 幂指型$u^v = e^{vlnu}$满足复合函数极限条件,所以$\lim e^\blacktriangle = e^{\lim\blacktriangle}$
		\item 针对{\bfseries$1^{\infty}$}型,极限尝试凑$\lim\limits_{\blacktriangle \to 0}(1+\blacktriangle)^{\frac{1}{\blacktriangle}}$
		\item 记住两个特殊的极限 {\bfseries 背诵直接使用} $\lim\limits_{x \to +\infty} x^{\frac{1}{x}}=1 、\lim\limits_{x \to 0^{+}} x^x = 1$
	\end{enumerate}

\begin{DerivativeExamples}
	\item 由极限值确定函数式中的参数
	\item 确定a,b使$\quan{1} \lim\limits_{x\to+\infty} \sqrt{2x^2+4x+1}-ax-b=0 \;\quan{2} \lim\limits_{x\to-\infty} \sqrt{2x^2+4x+1}-ax-b=0 $
	\item \refquestionsource{2000}{\Rmn{2}}若 $\lim_{x\to0}{\frac{sin{6x}+xf(x)}{x^3}}=0 $,则 $\lim_{x\to0}{\frac{6+f(x)}{x^2}}=\underline{\qquad} $
\end{DerivativeExamples}


\begin{thrmEqs}[泰勒]
	\item $e^x=1+x+\frac{1}{2!}x^2+\frac{1}{3!}x^3+\cdots+\frac{1}{n!}x^n+\circ(x^n)$
	\item $\ln(1+x)=x-\frac{1}{2}x^2+\frac{1}{3}x^3+\cdots+\frac{(-1)^{n-1}}{n}x^n+\circ(x^n)$
	\item $\sin{x}=x-\frac{1}{3!}x^3+\frac{1}{5!}x^5+\cdots+\frac{(-1)^{2n-1}}{(2n-1)!}x^{2n-1}+\circ(x^{2n-1})$
	\item $\cos{x}=1-\frac{1}{2!}x^2+\frac{1}{4!}x^4+\cdots+\frac{(-1)^{n}}{(2n)!}x^{2n}+\circ(x^{2n})$
	\item $f(x)=f(x_0)+\frac{f^\prime(x_0)}{1!}(x-x_0)+\frac{f^{\prime\prime}(x_0)}{2!}(x-x_0)^2+\cdots+\frac{f^{(n)}(x_0)}{n!}(x-x_0)^n+\circ(x^n)$fjslfjl
\end{thrmEqs}




	\begin{Theorems}
	这里是定理1.0
	\end{Theorems}
	假设这是一个问题	1111111111111fsflsfl
\begin{PROBLEMANSWERS}
	\begin{Analysis}
		这里是答案显示区
	\end{Analysis}
	\begin{Assess}
		这里是大众点评
	\end{Assess}
\end{PROBLEMANSWERS}

\begin{examples}
	$③\begin{matrix}\lim_{x\to\infty} (sin{\sqrt{x+1}}-sin{\sqrt{x}})\end{matrix}④\begin{matrix}\lim_{x\to{0}} \dfrac{\sqrt{1+x}-\sqrt{1-x}}{\sqrt[3]{1+x}-\sqrt[3]{1-x}}\end{matrix}$

\end{examples}


\begin{homeworks}
	\item \refquestionsource{$1997$}{ $\Rmn{3}$ }若$\lim\limits_{x\to0}{\dfrac{\sin{6x}+xf(x)}{x^3}}=0$, 则 $\lim_{x\to0}{\frac{6+f(x)}{x^2}}=$\fillblank{2}
\end{homeworks}



\ifnum \value{page}>1
	$4$
\fi
