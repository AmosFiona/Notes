\documentclass[no-math]{lecture}

\answer{1} % display answer
%\answer{0} % hide answer

\begin{document}
	\raggedbottom
	\abovedisplayshortskip = 4pt
	\belowdisplayshortskip = 4pt
	\abovedisplayskip 		 = 4pt
	\belowdisplayskip 		 = 4pt
	\frontmatter
		\tableofcontents
	\mainmatter
\newpage
	\chapter{极限}
	\quan{990}
	\hquan{88}
	\begin{Theorems}
	这里是定理1.0
	\end{Theorems}
%	\begin{tcolorbox}[assess]
%		djsllfalsn\\
%		ddd dangseeeee\\
%		d
%	\end{tcolorbox}
	假设这是一个问题	1111111111111fsflsfl
\begin{PROBLEMANSWERS}
	\begin{Analysis}
		这里是答案显示区
	\end{Analysis}
	\begin{Assess}
		这里是大众点评
	\end{Assess}
\end{PROBLEMANSWERS}

	\section{七个未定式}

\subsection{七种未定式}

\begin{LearnningObjectivesBox}[4]
	\begin{learningtargets}
		\item $\frac{0}{0}$ 型
		\item {\small $\dfrac{\infty}{\infty}$}型
		\item {\small ${\infty}-{\infty}$}型
		\item  ${\infty}\cdot{0}$ 型
		\item $1^{\infty} 、0^0 、{\infty^0} $ 型
	\end{learningtargets}
\end{LearnningObjectivesBox}
\begin{KeyandDifficultyBox}[4]
	\begin{learningkeys}
		\item[重点:] 泰勒公式、四则运算、洛必达法则
		\item[难点:] 泰勒公式
	\end{learningkeys}
\end{KeyandDifficultyBox}

\subsubsection{$\frac{0}{0}$ 型}
	\begin{enumerate}
		\item 无穷小相减$\Rightarrow$泰勒公式
		\item 遇见$\sqrt{\Box}$ 先有理化后观察
		\item 四则运算与洛必达法则配合
		\item 变限积分所表示的函数求极限采用洛必达法则\\
			\quan{1} 积分号里面含$x$拆项到积分号外\\
			\quan{2} $f$里含有$x$,需要用变量代换,将其变化到积分限中才可以用积分求导定理
	\end{enumerate}


\begin{examples}

	$③\begin{matrix}\lim_{x\to\infty} (sin{\sqrt{x+1}}-sin{\sqrt{x}})\end{matrix}④\begin{matrix}\lim_{x\to{0}} \dfrac{\sqrt{1+x}-\sqrt{1-x}}{\sqrt[3]{1+x}-\sqrt[3]{1-x}}\end{matrix}$

	$\begin{matrix}\lim_{x\to{0}} \frac{\arctan{x}-\sin{x}}{x^3}\end{matrix}$

	$\begin{matrix}\lim_{x\to{0}} \frac{e^x-\sin{x}-1}{1-\sqrt{1-x^2}}\end{matrix}$

	$\begin{matrix}\lim_{x\to{0}} \frac{\arcsin{x}-\arctan{x}}{\tan{x}-\sin{x}}\end{matrix}$

	$ \begin{matrix}\lim_{x\to{0}} \frac{\sqrt{1+2\tan{x}}-x-1}{\sin{x}-\ln(1+x)}\end{matrix}$

		 $ ②\begin{matrix}\lim_{x\to{0}} \big(\frac{1}{x^2}-\frac{1}{\sin^2{x}})\end{matrix}$

			$	 \begin{matrix}\lim_{x\to+\infty} x^2\big(e^{\frac{1}{x^2}+\frac{1}{x}}-e^{\frac{1}{x}}\big)\end{matrix} $
			$	  \begin{matrix}\lim_{x\to{0^+}} \big(\frac{1}{\sqrt{x}}\big)^{\tan{x}}\end{matrix}$

			$		 \begin{matrix}\lim_{x\to\infty} x\bigg [{\sin{\ln(1+\frac{3}{x})}-\sin{\ln(1+\frac{1}{x}})}\bigg ] \end{matrix}$

		\end{examples}
\begin{Homework}
	sfslfs  \hl{将是}一个难以理解的图案和设计
\end{Homework}

sfslfs  \hl{将是}一个难以理解的图案和设计

\begin{homeworks}
\item \refquestionsource{$1997$}{ 数学$\uppercase\expandafter{\romannumeral3}$ }若$\lim\limits_{x\to0}{\dfrac{\sin{6x}+xf(x)}{x^3}}=0$, 则 $\lim_{x\to0}{\frac{6+f(x)}{x^2}}=$\fillblank{2}
\end{homeworks}


\begin{Theorems}
	2这里是定理
	\end{Theorems}

	\begin{Theorems}
	3这里是定理
	\end{Theorems}


\ifnum \value{page}>1
	$4$
\fi



	%\section{七个未定式}
	\section{中值定理0}
		\subsubsection{谁说的准的事情呢}
\begin{DerivativeExamples}
	\item  sdfsdjl
	\item  sadf
	\end{DerivativeExamples}
		\subsubsection{的打饭的师傅是法律是解放军是否}
	\section{中值定理1}
	\section{中值定理2}
	\section{中值定理3}
	\section{中值定理4}
\newpage
	\chapter{微分方程}
	\section{中值定理5}
	\section{中值定理6}
	\section{中值定理7}
	\section{中值定理8}
	\section{中值定理9}
	\section{中值定理10}
\end{document}
