\documentclass[no-math]{course}

%\setCJKmainfont{StandardArial}
%\setCJKsansfont{StandardArial}
%\setCJKmonofont{StandardArial}

\begin{document}

	\raggedbottom
	\abovedisplayshortskip=5pt
	\belowdisplayshortskip=5pt
	\abovedisplayskip=5pt
	\belowdisplayskip=5pt
	\frontmatter
		\tableofcontents
	\mainmatter
	\chapter{极限}
	\section{七个未定式}
	\chapter{中值定理}
	\begin{paracol}{2}

	\switchcolumn
		\subsection{知识回顾}
			我们在初中阶段学习了导数基本定义
		\begin{empheq}
		[box=\eqmybox]{align}
			\lim_{x\to+\infty} \sqrt{2x^2+4x+1}-ax-b=0 \\
			\lim_{x\to-\infty} \sqrt{2x^2+4x+1}-ax-b=0
		\end{empheq}

	\begin{examples}
		若$\lim\limits_{x\to0}{\frac{\sin{6x}+xf(x)}{x^3}}=0$,则$\lim_{x\to0}{\frac{6+f(x)}{x^2}}=\underline{\qquad}$
	\end{examples}

	\begin{analysis}
		\marginpar{
			\begin{postil}
			分段计算了发生口角法律上了飞机上房间里睡觉发牢骚了发生都是看见了发生了!
			\end{postil}
		}
			分段计算了发生口角法律上了飞机上房间里睡觉发牢骚了发生都是看见了发生了!
			分段计算了发生口角法律上了飞机上房间里睡觉发牢骚了发生都是看见了发生了!
			分段计算了发生口角法律上了飞机上房间里睡觉发牢骚了发生都是看见了发生了!
	dlfdfjsljflsdfkdnfdfldfljl \\
	dfdsfsfs \\
	dddss\\
	dangseeeee
	\end{analysis}

\end{paracol}
	%【例45】确定a,b使;①\begin{matrix}\lim_{x\to+\infty} \sqrt{2x^2+4x+1}-ax-b=0\end{matrix} \;②\begin{matrix}\lim_{x\to-\infty} \sqrt{2x^2+4x+1}-ax-b=0\end{matrix}

	\begin{Exercise}
	\begin{exercises}
		\item for
		\item labelwidth
		\item and
	\end{exercises}
\end{Exercise}

\end{document}
